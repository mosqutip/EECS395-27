\documentclass{acm_proc_article-sp}
\usepackage{listings}
\usepackage{graphics} 
\usepackage{multirow}
\usepackage{url}

\begin{document}

\title{Predicting Security Vulnerabilities in Software Components through Code 
and Meta-level Analysis}

\numberofauthors{4}
\author{
\alignauthor
Roca He\\
    \affaddr{Zhejiang University}\\
    \email{heboyuan@gmail.com}
\alignauthor
Josiah Matlack\\
    \affaddr{Northwestern University}\\
    \email{jmatlack\\@u.northwestern.edu}
\and
\alignauthor
Chao Shi\\
    \affaddr{Northwestern University}\\
    \email{chaoshi2012\\@u.northwestern.edu}
\alignauthor
Maciek Swiech\\
    \affaddr{Northwestern University}\\
    \email{maciejswiech2007\\@u.northwestern.edu}
}

\maketitle
\begin{abstract}
In this paper, we present our research and findings toward an automated 
vulnerability analysis tool, which uses existing databases and code change 
logs, as well as PDG construction and dependency analysis, to predict 
undiscovered vulnerabilities in target software. Through semantic-level 
understanding of code, our aim is to apply a deeper analysis than provided by 
tools currently on the market.
\end{abstract}

\section{Introduction}
Software vulnerabilities are security-related bugs software that may be 
exploited by threats. Vulnerabilities are known to cost millions or billions of 
dollars to the affected vendors and users (Telang and Wattal 2005). Numerous 
methods have been used in practice to detect and eliminate vulnerabilities from 
software. These include manual code auditing, automatic testing, and static 
analysis. Despite years of research on eliminating vulnerabilities however, the 
number of vulnerabilities is still on the rise (Anon.).

In this proposal, we take a different approach towards vulnerability 
mitigation. With several years of software development and maintenance, we 
believe there is enough data to guide prediction of vulnerabilities in a 
meaningful manner. Intuitively and at a high-level perspective, studying code 
that was found to be vulnerable in the past may help to predict currently 
unknown vulnerabilities. That is the code in a previously vulnerable site and 
that in another site with a similar but yet unknown vulnerability is similar 
in some sense. Code similarity for such purposes may be measured in the forms 
of the dependencies of the components or pieces of code, or the system calls 
used, or other specific features such as checks of specific error numbers, and 
so on.

For this project, we will apply modern machine learning and data mining 
techniques on several security-focused features such as security advisories, 
alerts, and release notes, and also features from static analysis that we will 
develop. Specifically, for static analysis, we believe that control-flow and 
data-flow based analysis that reveals semantic-level artifacts of the code may 
be especially helpful. Furthermore, we will also use the version control 
history to obtain patterns relating different security advisories and alerts, 
results from static analysis, and possibly even coding styles. Previous work 
(Neuhaus et al. 2007; Neuhaus and Zimmermann 2009) has shown success at using 
component dependencies for predicting vulnerabilities. We will therefore also 
use dependencies amongst components as features in our system. While the 
security advisories and alerts and version control data are the meta-level 
information that hint at the pieces of code at which to focus, the code-level 
information obtained from dependency analysis and static analysis will provide 
us with a further understanding of code fragments that may be specially 
vulnerable.

Apart from the use of various features as described above, we will also use 
multiple machine learning and data mining techniques (including decision trees, 
neural networks, support vector machines, and association mining) and finally 
develop ensemble models based on 1these individual techniques to derive the 
best results. We expect to obtain from our model actionable insights that will 
speed up vulnerability discovery by directing related effort on specific areas 
in the code.

\section{Work}
As we have briefly discussed in earlier sections, our system is divided into 
three main parts. These parts are the collection of vulnerability patch 
metadata, dependency analysis of affected code, and the analysis of program 
dependence graphs for affected code. This section describes each of these 
techniques in greater detail.

\begin{figure*}
\centering
\epsfig{file=fig1.png, height=4in, width=6in}
\caption{A diagram of the proposed system.}
\end{figure*}

\subsection{Metadata Collection}
In order to collect data on common vulnerabilities and their corresponding 
fixes, we surveyed the Mozilla Foundation Security Advisories (MFSA) database. 
This is a database that includes all known vulnerabilities that affect Mozilla 
products and instructions on how users can protect themselves. Each commit to 
the code base has a BugID, and Each report has a unique MFSA ID associated with 
one or more Bug IDs.

We downloaded the code base using Mercurial, and dumped all the corresponding 
change logs. Then, each MFSA report was mapped to change sets or source code 
files using the Bug ID in the description of the change log. We used this 
database of information to find vulnerabilities. 

\subsection{Dependency Analysis}
First, we need to statically extract all the imported libraries and function 
calls from the source files. To do this, less effort is needed than completely 
parse the source code and build the abstract syntax tree as expected in the 
program dependency analysis.

To extract the libraries imported by the source file, the parser will read the 
source code line by line and match the pattern of \#include "xxx". The headers 
from the standard libraries in the format of \#include "xxx" are not what we 
are interested. Besides, the C and C++ has compilation dependencies such as "\#
ifdef" which makes the dependency analysis much more complex. For the time 
being, we are not considering any of these macros.

For the function call extraction, we are matching the patter of "id (id, id, 
...)". Taking the idea from Neauhaus paper, the reserved identifiers such as 
"if" or "for" are excluded. Besides, we are also trying to extract the 
functions defined and implemented by a single source file. Therefore, other 
source files calling a specific function can locate which source file the 
function is defined and implemented. Since the code base is pretty large and 
there are many identical naming of files, this method is based on the best 
effort and is by no means completely accurate.

We are examining the code base of Mozilla, which contains altogether 13740 C or 
C++ source files. The process is pretty fast and is terminated within 4 minutes.

After mining the Mozilla CVS and security advisories, we find altogether 256 
vulnerabilities and 556 vulnerable components out of 13740 components. The next 
target is to examine what combination of the imported libraries and function 
calls will have high potential to cause vulnerability in the source code.

Different from the practice in Neauhaus paper, we are considering the set of 
the libraries as a single feature. If a file imported N libraries denoted as 
set S, then it will have 2\^N-1 features, which comprises the super set of S 
except the empty set. For example, if a source file imported headers A.h, B.h, 
C.h. Then the source file will have 7 features, namely {A.h},{B.h}, {C.h}, 
{A.h, B.h}, {A.h, C.h}, {B.h, C.h} and {A.h, B.h, C.h}. However, we notice that 
the number of the features will grow up exponentially as the size of library 
headers grows. To deal with this problem, we require that the number of the 
library headers a feature contains to be 3.

Next, we will have to filter the features; the feature size is still large even 
we limit the number of libraries in the feature. Some combination of the 
headers will never appear in the Mozilla code base or they simply appear in the 
non-vulnerable components. Therefore, we require that the feature should appear 
at least in 3% of the vulnerable components, i.e. appearing at least 18 
vulnerable source files. After this filtering, we will have 729 such features.

These features are appearing quite rarely in the non-vulnerable components but 
they do appear. We found 3956 non-vulnerable components out of 13740 which see 
at least one of these 729 features. The total appearing of features among these 
3956 non-vulnerable components is 21700 with each non-vulnerable component 
seeing around 5.5 features individually. On the contrary, the 556 vulnerable 
source files are altogether seeing 15826 appearing counts, with each one seeing 
28.5 on average. Some interesting findings about the features are summarized in 
table below.

\begin{table}
\centering
\caption{Survey of vulnerable components}
\begin{tabular}{|c|c|l|} \hline
Feature&Vulnerable&Non-vulnerable\\ \hline
nslviewManager.h \\ nslDocument.h \\ nsFocusManager.h & 16 & 1 \\ \hline
nslPresShell.h \\ nsFrameManager.h \\ nsLayoutUtils.h & 15 & 3 \\ \hline
jsXXX.h \\ Marking.h & 20 & 1 \\ \hline
\end{tabular}
\end{table}

We see that some of these libraries combined can produce some insights. For 
example, the first combination might be related to graphic components with the 
focusing capability. The second combination might be vulnerabilities coming 
from the frame components which have layout features. The third combination is 
seen in a lot of places, with one of the headers is Marking.h and the other two 
are all JavaScript headers. Although the size of instances is not large enough, 
the majority of this combined features are from the vulnerable components.

For mining on the libraries and the function calls, the python scripts are 
written and the information is stored in the database server run by Mysql. Some 
of the data process part is done manually by typing sql commands to fabricate 
the training samples and validation samples. For the machine learning 
algorithm, the library of LIBSVM is used.

\subsection{Program Dependence Graph Analysis}
In predicting software vulnerabilities, we require some basis of comparison 
between vulnerable code that has subsequently been patched, and the code being 
tested for analysis. The main technique we employed for this comparison was the 
construction and comparison of program dependence graphs (PDG). A PDG is a 
directed graph with vertices representing the statements of a program and the 
edges representing data, control, and alias flows.

Initially, we attempted to construct PDGs using libclang, a library of Python 
bindings for the LLVM compiler. LLVM provides various utilities for program 
analysis of C++ code, including some rudimentary mechanisms for creating 
control flow graphs and abstract syntax trees. After one of our weekly progress 
meetings, our team and Professor Chen came to the conclusion that, even with 
the utilities provided by LLVM, analysis of C++ code would be prohibitively 
hard, especially considering the short time frame of this project. This stems 
from the variability in C++ libraries and huge number of language features that 
it provides, and was made abundantly clear after analysis of various networking 
vulnerabilities that had been discovered in the Android operating system.

After this point, we shifted our focus to analysis of C code. Toward this end, 
we discovered a tool for automated C language PDG construction, called Frama-C. 
Frama-C allows for the creation of PDGs, and outputs the information in both a 
text format and a visual representation created using graphviz. Using this 
tool, we were able to create PDGs for various C programs. In addition, we 
developed a Python script, adapted from our original work with libclang, that 
allowed for the parsing of the Frama-C output into a series of interconnected 
node objects. This script facilitated easier comparison of PDGs than was 
available with the original output from Frama-C (our original method of 
comparison was basically performed by hand).

\begin{figure*}
\centering
\epsfig{file=fig2.png, height=2in, width=7in}
\caption{A diagram of the PDG construction process.}
\end{figure*}

\section{Evaluation}
\subsection{Finding Vulnerability Examples}
We discovered many vulnerabilities through the use of the MFSA database. One 
example of such a vulnerability is the SSL certification validation 
vulnerability. Two options used by cURL in SSL verification are vulnerable to 
authentication errors if their values are set incorrectly. These values are \\
\\
CURLOPT\_SSL\_VERIFYPEER \\ 
CURLOPT\_SSL\_VERIFYHOST \\
\\
Setting these options to '0' ('false') or '1' ('true') wil skip certificate 
validation. The only valid option is '2'.

Another vulnerability found is also related to SSL certificates. In this, the 
gnutls\_certificate\_verify\_peers2 function in the GnuTLS library may return 0, 
indicated an unsuccessful validation. However, this return code is not checked 
for this value, so errors may go by without identification.

In addition to these errors, we also found a few simple buffer overrun errors, 
and a few miscellaneous XSS bugs. The results of the database scan are shown 
below.

\begin{table}
\centering
\caption{Database Mining Results}
\begin{tabular}{|c|c|l|} \hline
& C & C and C++ \\ \hline
Number of change sets & 404,650 & 404,650 \\ \hline
Number of C/C++ related bugs & 84,138 & 152,394 \\ \hline
Number of MFSA reports & 542 & 542 \\ \hline
Number of C/C++ vulnerabilities & 132 & 256 \\ \hline
\end{tabular}

\end{table}
\subsection{Dependency Analysis}
After getting the vulnerable components and the non-vulnerable components 
sharing a set of features, we will be able to use machine learning techniques 
to classify new instances. The machine learning techniques we are using is 
support vector machine. The soft margin support vector machine variation is 
used with the variable of C set to be 1 as standard.

Unfortunately, for this stage, we will not be able to predict new 
vulnerabilities. The reason is that (1) Different from other machine learning 
task, which is easy to generate new data, this approach is highly dependent on 
the code base. Unless there are new components added to the code base, there 
are no good samples to test our machine learning algorithm. (2) The label of 
the new components is hard to determine. There is no fast way of generating the 
vulnerability label for the new components (Hard to do bug or vulnerability 
proof) while the support vector machine classifier can be very fast and work in 
large data set. This unbalance between the data generation and learning 
algorithm further limits our application.

Based on these limitations, we will use k-fold cross-validation to show at 
least that the dependency relationship can be learned to some extent. K is set 
to be 3 in this case, with the ratio between the training set and validation 
set to be 2:1.

After our random evaluation for 1000 times, the average false positive rate is 
51.7%. The rate is high in the absolute view but is acceptable in the security 
field. Of all the components we believe that is vulnerable, only half of them 
are vulnerable in the real case. One belief which can relieve us is that some 
of the components are vulnerable but are considered non-vulnerable for the time 
being because people have not discovered them. That means the false positive 
rate is an upper bound and the real value should be lower than this.

On the other hand, the false negative rate is only 0.37%, which is really 
small. Nearly all the vulnerable components can be successfully captured using 
the dependency analysis approach.

\subsection{Lynx Libraries}
Unfortunately, we were unable to meaningfully test the Lynx libraries using 
our PDG analysis, owing to multiple problems between the Frama-C framework and 
the Lynx library code. Frama-C appears to have issues with header files not 
included in the same directory as the file being analyzed. Frama-C uses gcc to 
pre-process the code, and so requires all symbols and libraries and calls to be 
explicated. Frama-C seems to be unable to deal with the complexity of elements 
such as GCC builtins, and C library code. We tried for many hours to debug 
these problems, but were unable to resolve errors such as:

\begin{lstlisting}
/usr/include/bits/byteswap.h:47:[kernel]
warning: Calling undeclared function
\_\_builtin\_bswap32. Old style K&R code?
\end{lstlisting}

In this case, Frama-C is claiming that there are errors in standard C
library code. Despite our best research efforts, these issues remain 
unresolved. Hopefully, with future input, these issues will be cleared up.

\subsection{Test Programs}
Our PDG construction was run on two small test programs, which can be found in 
the repository under data-ctrl.c and pdg-ex.c. The corresponding PDG outputs 
can be found in small-out and big-out. These graphs are not presented here due 
to the sheer size. Comparison analysis was not run on these small programs, 
because we have no existing corpus of possible vulnerabilities with which to 
compare them. However, the comparison tool can still be run on these files, if 
there is interest.

These small files were encouraging, as they provided proof-of-concept for the 
PDG generation, and also allowed for some calibration and testing of further 
scripts and routines that we attempted. Forming a playground of sorts, these 
programs allowed for rapid test iteration and bug fixes in our framework code. 
However, as test programs, and with the lack of meaningful comparison, these 
programs did not provide us with much research information.

\section{Related Research}
Attempts to quantify software reliability have been made since the 70s (Musa, 
Iannino, and Okumoto 1987). Many models have been developed that try to relate 
software faults to the age of the software, code complexity, programmer 
experience, organizational maturity, and so on (Takahashi and Kamayachi 1985; 
Engel and Last 2007). Security vulnerabilities are a special type of faults and 
may need special considerations for reliable estimation and prediction. It is 
more desirable to incorporate independent variables that have what are believed 
to be strong physical relationships to security vulnerability, and not 
variables that are more generic in nature.

Alzhami et al. (O. Alhazmi, Malaiya, and Ray 2005; O. H. Alhazmi, Malaiya, and 
Ray 2007) have developed regression models around the vulnerabilities of major 
software, such as different releases of Windows and Red Hat Linux. These models 
may be able to predict the rate at which vulnerabilities are discovered in 
general but cannot pin point the components that may be more vulnerable than 
others. Hence, these models provide little actionable information. As discussed 
earlier, previous research has empirically showed that dependencies amongst 
software components have good vulnerability-prediction capability (Neuhaus et 
al. 2007; Neuhaus and Zimmermann 2009). Because of previous success of 
dependency-based prediction, we will actively explore it further in this 
proposal as well.

CP-Miner is a tool for finding copy-paste bugs in source code. This project was 
eventually turned into the commercial system Pattern Insight, albeit with some 
algorithmic changes and improvements. CP-Miner looks only for copy-pasted code 
similarities in source code. The tool functions by using tokenization in what 
the authors call a "code-based" system. This differs from other systems, 
including our own, that may use parse trees (syntax based), string-based 
systems, and semantic systems. Unlike CP-Miner, our project relies on semantic 
understanding of the code in addition to syntactic analysis. The authors of the 
paper discussing CP-Miner add that it is very difficult to find copy-paste bugs 
using static and dynamic analysis, mostly because the same exploits must be 
rediscovered on each incarnation. Theoretically, our system would solve this 
problem, however the time complexity would be much greater than for the same 
problem using CP-Miner. CP-Miner allows for quick discovery and comparison of 
all copy-paste errors. CP-Miner works by using an algorithm called subsequence 
mining. This algorithm maps similar-looking code to the same index value, in 
lines or in blocks. Then, it looks for groupings in other parts of the code 
that are similar to previously indexed blocks. Finally, it compares these 
segments, and sees if variable changes and bugs have propagated across the two 
blocks by using a "change" ratio. By comparing this ratio to a threshold, bugs 
are identified. Like the next system, ReDeBug, this is a statistical analysis, 
and very different in usage and implementation than our proposal.

Pattern Insight, an industry tool based on CP-Miner with algorithmic 
improvements, works approximately in the same way. However, the system relies 
on a larger database of comparison information, much like the DejaVu system 
which will be discussed later.

ReDeBug is a similar system to CP-Miner, but has some novel ideas. ReDeBug 
looks for "code clones", which by the authors' definition, are similar copies 
of code blocks across source files. In this way, it is much like CP-Miner, 
however, ReDeBug is a syntax-based system. Once again, the lack of semantic 
understanding, and the lack of expansion beyond copy-pasted code makes this 
system limited in a way that our project is not. ReDeBug operates by removing 
whitespace, special characters (like braces), and tokenizing source code lines. 
A sample of n tokens is taken, and compared against subsequent samples in a 
source file. Another parameter, theta, is used as a required threshold for 
similarity between code blocks. The combination of these parameters is used 
such that ReDeBug is a statistical analysis, rather than lexicographical 
analysis. To save on space and complexity constraints, the comparison vectors 
are stored in Bloom filters, which are then compared via unions. This analysis 
is again very different from our proposal, as semantic understanding of code is 
not achieved. In addition, the only discussion of security vulnerabilities 
found by ReDeBug, rather than non-security related bugs, uses the same 
clone-detection mechanism, so no additional complexity or analysis is added for 
security holes.

DejaVu is another tool for clone analysis, and acts basically as a superset of 
CP-Miner (in terms of it effectiveness in detecting bugs). It goes without 
saying the DejaVu is also dissimilar to our proposal, since once again semantic 
understanding is not achieved. DejaVu, while detecting a superset of CP-Miner 
bugs, is implemented in a very similar fashion to ReDeBug. The system forms an 
abstract syntax tree (AST), and uses an unstructured sequence of tokens 
(usually 30 to 50 tokens in length) in a sliding window on the AST to look for 
code clones. DejaVu applies a filtering technique using minimal textual 
similarity to cut down on the number of detected bugs, and eliminate benign 
detections. Like CP-Miner, the end goal of DejaVu is to find inconsistent 
changes across code clones. However, as can be seen by the use of the AST and a 
sliding window of tokens, the analysis performed is purely semantic, and, while 
in-depth, fails to reach the semantic understanding of our proposed use of PDGs 
and deep code analysis.

The last of the major systems is DECKARD. In a recurring theme, DECKARD is a 
clone detection system, using a novel algorithm for clone detection in an AST. 
Again, semantic analysis is not achieved, and the authors specifically point 
out that program data graph (PDG) construction is avoided in this system. The 
authors also mention the use of PDGs in other systems, but upon reviewing these 
references, we could find no uses of PDGs in a security- or bug-discovery-
related context. The implementation details of DECKARD require extensive 
knowledge of complex mathematical formulas, which is beyond the scope of this 
analysis. However, it suffices to say that DECKARD, while complex, still does 
not take the semantic and deep-level understanding of code that our proposed 
project will.

The remaining systems, as aforementioned, are CCFinder, MOSS, and SYDIT, which 
are either referenced by the previous papers, found by independent research, or 
both. Nothing new is added by these systems, so we can safely say our project 
proposal is unique to the security analysis field (although the possibility of 
a similar system exists, we find it extremely unlikely). Without expounding 
extreme detail, we now summarize these systems.

CCFinder uses a tokenization and sliding window technique very close to ReDeBug 
and DECKARD. In fact, this system is so similar to ReDeBug that it could be 
seen as a spiritual clone.

MOSS is a plagiarism-detection tool used by Stanford University and others to 
detect copied code between student submissions. No implementation details could 
be found regarding this system, likely due to its purpose. However, evidence 
suggests that MOSS uses syntax-based clone detection, much like many of the 
aforementioned systems, and thus adds no novelty to our research.

Finally SYDIT is a system very close to DECKARD, and in fact, mentioned by the 
authors of the latter. It utilizes AST construction and contextual awareness to 
look for code "transformations", as coined by the authors. These 
transformations are spiritually the same as code clones, albeit with a 
different nomenclature.

This concludes our survey of previous and related work in the context of code 
analysis for security flaws. As evidenced by our research, our system would 
provide a unique and novel approach to source code analysis, and, while other 
systems have similar goals and share similar implementation mechanisms, none 
match the deep, semantic analysis of code that we wish to achieve. In addition, 
the use of PDGs seems fairly rare in current systems, which sets our system 
apart from others. Other systems use ASTs and database mining for code 
analysis, which we also plan to implement, but our combination of these two 
mechanisms, and the novel PDG construction make our system stand apart from 
others.

Not too much related work can be found about dependency analysis. Neauhaus 
(2007) examined the codebase of Mozilla and used both function calls and 
imported libraries as features to classify the vulnerable components. Neauhaus 
(2009) is an extension of their previous work, they used the similar machine 
learning techniques and apply it to the analysis of vulnerable software 
packages. Most of the work of this project is borrowing the idea of Neauhaus 
(2007).

\section{Presentation Comments}
During our final presentation, it was brought up that PDG comparison is a very 
difficult problem. Indeed this is true, and is a major motivation for the move 
from C++ to C code. As evidenced by our future work section, this is a 
continuing problem that requires additional work. Again, given the short time 
period we had for implementation, we were unable to produce a high-quality 
comparator. However, although difficult, we believe this comparison is possible
, with a high degree of accuracy, given more effort.

\section{Conclusion}
In this paper, we have presented our system for predicting software 
vulnerabilities using a combination of PDG construction, dependency analysis, 
and comparison to mined vulnerability databases. We have exposed components for 
each of these features in turn, showing how each is implemented and what 
purpose it fulfills for the system as a whole. We have also explored how other 
researchers have attempted similar analyses, but have failed to produce the 
level of semantic understanding and code analysis that we have aimed for. 
Finally, we have discussed the shortcomings of our system, what remains to be 
implemented, and how these issues can be overcome. In short, we have provided 
the basis for an automated, semantic-level vulnerability detector, which 
leverages existing code bases to predict vulnerabilities in new and untested 
software. It is our hope that our work can be extended, as per our 
recommendations, into a functional vulnerability analysis tool.

\section{References}
1 CCFinderX. \\
http://www.ccfinder.net/doc/10.2/en/whats.html.

2 Jiyong Jang, Abeer Agrawal, and David Brumley. "ReDeBug: Finding Unpatched 
Code Clones in Entire OS Distributions."

3 LIBSVM, http://www.csie.ntu.edu.tw/~cjlin/libsvm/

4 Lingxiao Jiang, Ghassan Misherghi, Zhendong Su, and Stephane Glondu. 
"DECKARD: Scalable and Accurate Tree-based Detection of Code Clones."

5 Mark Gabel, Junfeng Yang, Yuan Yu, Moises Goldszmidt, and Zhendong Su. 
Scalable and systematic detection of buggy inconsistencies in source code. In 
Proceedings of the ACM International Conference on Object Oriented Programming 
Systems Languages and Applications, 2010.

6 "Moss: A System for Detecting Software Plagiarism." \\
http://theory.stanford.edu/~aiken/moss/

7 Na Meng, Miryung Kim, and Kathryn S. McKinley. "Sydit: Creating and Applying 
a Program Transformation from an Example."

8 Neuhaus, Stephan, and Thomas Zimmermann. 2009. "The beauty and the beast: 
vulnerabilities in red hat's packages."

9 Neuhaus, Stephan, Thomas Zimmermann, Christian Holler, and Andreas Zeller. 
2007. "Predicting vulnerable software components." In Proceedings of the 14th 
ACM conference on Computer and communications security, 529-540.

10 Zhenmin Li, Shan Lu, Suvda Myagmar and Yuanyuan Zhou. "CP-Miner: A Tool for 
Finding Copy-paste and Related Bugs in Operating System Code." \\ 
http://theory.stanford.edu/~aiken/moss/

%
% The following two commands are all you need in the
% initial runs of your .tex file to
% produce the bibliography for the citations in your paper.
\bibliographystyle{abbrv}
\bibliography{sigproc}  % sigproc.bib is the name of the Bibliography
% You must have a proper ".bib" file
%  and remember to run:
% latex bibtex latex latex
% to resolve all references
%
% ACM needs 'a single self-contained file'!
%
\appendix

\section{README}
\subsection{Code Location}
https://github.com/mosqutip/EECS395-27

\subsection{How to Run the Code}
Frama-C: frama-c/frama-c -pdg-print -pdg -cpp-command 'gcc -C -E <options>' 
<file1.c> <file2.c> ... \\
parser/comparator: ./comparator.py <input\_pdg\_file>

\balancecolumns
\end{document}
